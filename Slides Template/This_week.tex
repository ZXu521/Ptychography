%============================================================
%     Group Meeting Report — Computational Imaging (English)
%============================================================
\documentclass[15pt]{beamer}
\makeatletter
\makeatother

%======================== packages & settings ========================
\usepackage{amsmath, pifont, bookmark, tabularx, booktabs, adjustbox, graphicx, caption, subcaption, listings, color, tikz}
\usetikzlibrary{arrows.meta,positioning,shapes.geometric,shapes.misc,fit,calc,decorations.pathreplacing}
\usepackage{hyperref}
\hypersetup{colorlinks=true,linkcolor=blue,urlcolor=blue}

%======================== theme & fonts ========================
\usetheme{Montpellier}
\renewcommand{\familydefault}{\sfdefault}

% TOC numbered
\setbeamertemplate{section in toc}[sections numbered]
\setbeamertemplate{subsection in toc}[subsections numbered]

% footer
\setbeamerfont{footline}{size=\tiny}
\setbeamertemplate{footline}{
  \leavevmode
  \hfill \insertframenumber{} / \inserttotalframenumber \hspace{2ex}\vskip2pt}

% title fonts
\setbeamerfont{title}{size=\fontsize{20pt}{24pt}\selectfont}
\setbeamerfont{subtitle}{size=\fontsize{12pt}{14.4pt}\selectfont}
\setbeamerfont{author}{size=\fontsize{10pt}{12pt}\selectfont}
\setbeamerfont{date}{size=\fontsize{10pt}{12pt}\selectfont}

%======================== macros ========================
\newcommand{\Obj}{\mathrm{O}}
\newcommand{\Probe}{\mathrm{P}}
\newcommand{\Exit}{\psi}
\newcommand{\uu}{\mathbf{u}}
\newcommand{\rr}{\mathbf{r}}
\newcommand{\Rj}{\mathbf{R}_j}
\newcommand{\kk}{\mathbf{k}}

\tikzset{
  >=Stealth,
  line/.style      = {->, thick},
  block/.style     = {rectangle, rounded corners, draw, align=left, thick, inner sep=6pt, fill=white},
  io/.style        = {trapezium, trapezium left angle=70, trapezium right angle=110, draw, thick, align=left, inner sep=6pt, fill=white},
  smallblock/.style= {rectangle, rounded corners, draw, align=left, thick, inner sep=4pt, fill=white, font=\small},
  loopbox/.style   = {rectangle, rounded corners, draw, dashed, thick, inner sep=8pt},
  note/.style      = {rectangle, draw, align=left, inner sep=4pt, fill=white, font=\scriptsize}
}

%======================== basic info ========================
\title{Weekly Progress Report}
\author{Zihan Xu}
\date{\today}

%============================================================
\begin{document}

%---------------- title ----------------
\begin{frame}[plain]
  \titlepage
\end{frame}
\addtocounter{framenumber}{-1}

%---------------- toc ----------------
\begin{frame}[plain]{Outline}
  \tableofcontents[sectionstyle=show, subsectionstyle=hide]
\end{frame}
\addtocounter{framenumber}{-1}

%============================================================
\section{Weekly Progress Overview}
%============================================================
\begin{frame}{Summary of This Week}
\footnotesize
\begin{itemize}
  \item Completed systematic understanding of \\Projection-based vs Optimization-based algorithms.
  \item Studied Maximum-Likelihood refinement for CDI (Thibault).
  \item Studied IterativeLeast-squares refinement for ML-Ptycho (PSI).
  \item Read "Automatic Parameter Selection for Electron Ptychography via Bayesian Optimization".
  \item Implemented Zhen Chen's Mixed-state ptychography code in Matlab.
\end{itemize}

\end{frame}

%============================================================
\section{Code Reproduction}
%============================================================
\begin{frame}{Reproduction of Chen Zhen et al. (2020) Results}
\footnotesize

\begin{center}\includegraphics[width=0.8\linewidth]{image1.jpg} 

\end{center}

\end{frame}

\begin{frame}{Reproduction of Chen Zhen et al. (2020) Results}
\footnotesize

\begin{center}\includegraphics[width=0.9\linewidth]{image2.png} 

\end{center}

\end{frame}

\begin{frame}{Reproduction of Chen Zhen et al. (2020) Results}
\footnotesize

\begin{center}\includegraphics[width=0.9\linewidth]{image3.png} 

\end{center}

\end{frame}
\begin{frame}{Reproduction of Chen Zhen et al. (2020) Results}
\footnotesize

\begin{center}\includegraphics[width=0.9\linewidth]{image4.png} 

\end{center}

\end{frame}
\begin{frame}{Reproduction of Chen Zhen et al. (2020) Results}
\footnotesize

\begin{center}

\includegraphics[width=0.7\linewidth]{probe_mag_Niter2000.jpg} 
\vspace{2mm}
\includegraphics[width=0.7\linewidth]{probe_Niter2000.jpg}

\end{center}

\end{frame}
\begin{frame}{Reproduction of Chen Zhen et al. (2020) Results}
\footnotesize

\begin{center}\includegraphics[width=0.3\linewidth]{obj_phase_roi_Niter1200.jpg} 
  \vspace{2mm}
\includegraphics[width=0.6\linewidth]{ZhenChen_exp.png} 
\end{center}

\end{frame}

%============================================================
\section{Algorithmic Philosophy: Projection vs Optimization}
%============================================================

\begin{frame}{Two Fundamental Families of Algorithms}
\footnotesize
Based on \textit{projection vs optimization} taxonomy derived from the uploaded notes (projection vs optimization.md).

\begin{block}{Projection-Based (No Global Loss)}
\begin{itemize}
  \item ER, HIO, RAAR
  \item Difference Map (DM), HPR, DR
  \item PIE, ePIE, rPIE, AD-PIE, mPIE
  \item Geometric constraints: Modulus \textit{projection}, Support \textit{projection}
  \item No global objective; iteration is geometric
\end{itemize}
\end{block}

\begin{block}{Optimization-Based (Has Global Loss)}
\begin{itemize}
  \item Maximum-Likelihood (ML) ptychography
  \item Poisson/Gaussian likelihood models
  \item Wirtinger gradient descent, Conjugate Gradient
  \item True global minimization
\end{itemize}
\end{block}

\end{frame}

%------------------------------------------------------------
\begin{frame}{Projection Family in Detail}
\footnotesize

\textbf{1. Classical POCS}
\begin{itemize}
  \item Sequential projections onto constraint sets
  \item ER / HIO / RAAR
\end{itemize}

\textbf{2. Difference Map (DM)}
\begin{itemize}
  \item Projection + reflection operator
  \item Strong convergence properties
\end{itemize}

\textbf{3. PIE Family}
\begin{itemize}
  \item Only diffraction constraint uses true projection
  \item Object/probe updates = local least-squares corrections
  \item No global cost function
\end{itemize}

\end{frame}

%------------------------------------------------------------
\begin{frame}{Optimization Family in Detail}
\footnotesize

Using content from Maximum-likelihood refinement notes.

\textbf{1. Global Loss Function}
\begin{itemize}
\item Poisson: $L = -\sum(n\log I - I)$
\item Gaussian: $L = \sum (I-n)^2 /(2\sigma^2)$
\end{itemize}

\textbf{2. Wirtinger Calculus}
\begin{itemize}
  \item Gradient wrt complex conjugate variable
  \item Efficient and avoids full Hessian
\end{itemize}

\textbf{3. $\chi$-Term (Error Wave)}
\begin{itemize}
  \item Fourier-domain error field
  \item Backpropagated to real space
\end{itemize}

\textbf{4. Conjugate Gradient (CG)}
\begin{itemize}
  \item Approximates Newton without Hessian
  \item Polak–Ribière formula for conjugacy
\end{itemize}

\end{frame}

%============================================================
\section{Maximum-Likelihood Ptychography Theory}
%============================================================

\begin{frame}{Poisson PMF and Physical Meaning}
\footnotesize
From the uploaded ML notes.

\begin{itemize}
  \item Detector counts discrete electrons: Poisson statistics
  \item $p(n|I) = I^n e^{-I}/n!$
  \item High count: low noise; low count: high noise
  \item Negative log-likelihood gives the global cost
\end{itemize}

\begin{block}{Negative Log-Likelihood}
$L = -\sum_{jq}(n_{jq}\log I_{jq} - I_{jq})$
\end{block}

\end{frame}

%------------------------------------------------------------
\begin{frame}{$\chi$-Term and Wirtinger Gradient}
\footnotesize

\textbf{Definition:}
$$\tilde{\chi}_{jq} = \frac{\partial L}{\partial I_{jq}}\tilde{\psi}_{jq} = \left(1 - \frac{n_{jq}}{I_{jq}}\right)\tilde{\psi}_{jq}$$

\textbf{After inverse FFT:}
$$\chi_{jr} = \mathcal{F}^{-1}(\tilde{\chi})$$

Meaning:
\begin{itemize}
  \item Detector-plane residual
  \item Backpropagated to sample plane
  \item Drives gradient updates for object and probe
\end{itemize}

\end{frame}

%============================================================
\section{Conjugate Gradient (CG) for ML Reconstruction}
%============================================================

\begin{frame}{Why Steepest Descent is Slow}
\footnotesize
\begin{itemize}
  \item Negative gradient is the steepest direction locally
  \item But oscillates in long, curved valleys
  \item "Zig-zag" behavior slows convergence
\end{itemize}
\end{frame}

%------------------------------------------------------------
\begin{frame}{Conjugate Gradient Essentials}
\footnotesize

Search direction:
$$d^{(n)} = -g^{(n)} + \beta^{(n)}d^{(n-1)}$$

Polak–Ribière coefficient:
$$\beta^{(n)} = \frac{\langle g^{(n)}, g^{(n)} \rangle - \langle g^{(n)}, g^{(n-1)} \rangle}{\langle g^{(n-1)}, g^{(n-1)} \rangle}$$

\begin{itemize}
  \item Avoids zig-zag motion
  \item Uses curvature without storing Hessian
  \item Excellent for high-dimensional ptychography
\end{itemize}

\end{frame}



%============================================================
\section{Summary and Next Steps}
%============================================================

\begin{frame}{Summary}
\footnotesize
\begin{itemize}
  \item Built a unified understanding of projection vs optimization methods.
  \item Derived ML ptychography gradients from first principles.
  \item Understood the role of $\chi$ as gradient carrier.
  \item Learned why CG is the preferred optimizer.
  \item Understood why electron ptychography requires ML.
\end{itemize}
\end{frame}

%------------------------------------------------------------
\begin{frame}{Next Week Plan}
\footnotesize
\begin{itemize}
  \item immigrate MLPIE to Python.
  \item study least-squares solver theories and implement it in Python.
  \item Try position refinement in electron pty.
  \item Study multislice theory.
  \item Try adaptive-z in electron pty.
\end{itemize}
\end{frame}

%============================================================
\section*{Acknowledgement}
%============================================================

\begin{frame}[plain]
\centering
\vspace{1cm}
{\Huge Thank You!}\\[3mm]
{\large Questions and Discussion}
\end{frame}

\end{document}
