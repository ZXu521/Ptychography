%============================================================
%     Group Meeting Report Template — Computational Imaging
%============================================================
\documentclass[15pt]{beamer}
\makeatletter
\makeatother

%======================== packages & settings ========================
\usepackage{amsmath, pifont, bookmark, tabularx, booktabs, adjustbox, graphicx, caption, subcaption, listings, color, tikz}
\usetikzlibrary{arrows.meta,positioning,shapes.geometric,shapes.misc,fit,calc,decorations.pathreplacing}
\usepackage{hyperref}
\hypersetup{colorlinks=true,linkcolor=blue,urlcolor=blue}

%======================== theme & fonts ========================
\usetheme{Montpellier}
\renewcommand{\familydefault}{\sfdefault}

% TOC numbered
\setbeamertemplate{section in toc}[sections numbered]
\setbeamertemplate{subsection in toc}[subsections numbered]

% footer
\setbeamerfont{footline}{size=\tiny}
\setbeamertemplate{footline}{
  \leavevmode
  \hfill \insertframenumber{} / \inserttotalframenumber \hspace{2ex}\vskip2pt}

% title fonts
\setbeamerfont{title}{size=\fontsize{20pt}{24pt}\selectfont}
\setbeamerfont{subtitle}{size=\fontsize{12pt}{14.4pt}\selectfont}
\setbeamerfont{author}{size=\fontsize{10pt}{12pt}\selectfont}
\setbeamerfont{date}{size=\fontsize{10pt}{12pt}\selectfont}

%======================== macros ========================
\newcommand{\Obj}{\mathrm{O}}
\newcommand{\Probe}{\mathrm{P}}
\newcommand{\Exit}{\psi}
\newcommand{\uu}{\mathbf{u}}
\newcommand{\rr}{\mathbf{r}}
\newcommand{\Rj}{\mathbf{R}_j}
\newcommand{\kk}{\mathbf{k}}

\tikzset{
  >=Stealth,
  line/.style      = {->, thick},
  block/.style     = {rectangle, rounded corners, draw, align=left, thick, inner sep=6pt, fill=white},
  io/.style        = {trapezium, trapezium left angle=70, trapezium right angle=110, draw, thick, align=left, inner sep=6pt, fill=white},
  smallblock/.style= {rectangle, rounded corners, draw, align=left, thick, inner sep=4pt, fill=white, font=\small},
  loopbox/.style   = {rectangle, rounded corners, draw, dashed, thick, inner sep=8pt},
  note/.style      = {rectangle, draw, align=left, inner sep=4pt, fill=white, font=\scriptsize}
}

%======================== basic info ========================
\title{Weekly Research Report: Computational Ptychography}
\subtitle{Group Meeting Template}
\author{Zihan Xu}
\date{\today}

%============================================================
\begin{document}

%---------------- title ----------------
\begin{frame}[plain]
  \titlepage
\end{frame}
\addtocounter{framenumber}{-1}

%---------------- toc ----------------
\begin{frame}[plain]{Outline}
  \tableofcontents[sectionstyle=show, subsectionstyle=hide]
\end{frame}
\addtocounter{framenumber}{-1}

%============================================================
\section{Research Progress Overview}
%============================================================
\begin{frame}{Summary of This Week}
\footnotesize
\begin{itemize}
  \item Implemented new simulation for PIE / ePIE with modified probe update.
  \item Compared convergence curves across multiple $\beta$ values.
  \item Identified asymmetric phase artifacts at edges.
  \item Prepared new results for potential conference submission.
\end{itemize}

\vspace{2mm}
\begin{block}{Key Results}
\centering
% \includegraphics[width=0.6\linewidth]{placeholder_result.png}
{\scriptsize Example: Reconstructed amplitude and phase (placeholder).}
\end{block}
\end{frame}

%============================================================
\section{Algorithmic Development}
%============================================================
\begin{frame}[fragile]{Algorithmic Updates}
\footnotesize
\textbf{Focus:} Explore new hybrid update strategy between ePIE and mPIE.

\begin{itemize}
  \item Introduced momentum-like correction term for object update.
  \item Modified probe normalization for stability.
  \item Implemented adaptive $\beta$ scheduling to accelerate convergence.
\end{itemize}

\vspace{2mm}
\begin{exampleblock}{Current Implementation (Python snippet)}
\scriptsize
\begin{lstlisting}[language=Python]
O_next = O_curr + beta * P_conj / (|P|^2 + alpha) * (Psi_corr - Psi_guess)
momentum = gamma * (O_next - O_prev)
O_next += momentum
\end{lstlisting}
\end{exampleblock}
\end{frame}


%============================================================
\section{Experimental / Simulation Results}
%============================================================
\begin{frame}{Simulation Results}
\footnotesize
\begin{columns}[T,totalwidth=\textwidth]
  \begin{column}{0.5\textwidth}
    \centering
    % \includegraphics[width=\linewidth]{placeholder_object.png}
    \scriptsize Reconstructed Object (placeholder)
  \end{column}
  \begin{column}{0.5\textwidth}
    \centering
    % \includegraphics[width=\linewidth]{placeholder_probe.png}
    \scriptsize Retrieved Probe (placeholder)
  \end{column}
\end{columns}

\vspace{3mm}
\scriptsize
\textbf{Observation:}  
Stable convergence within first 100 iterations.  
Residual phase noise concentrated at boundary due to overlap deficiency.
\end{frame}

%============================================================
\section{Discussion and Challenges}
%============================================================
\begin{frame}{Issues and Analysis}
\footnotesize
\begin{itemize}
  \item \textbf{Asymmetric edge artifacts:} due to raster scan bias.
  \item \textbf{Probe drift simulation:} not yet stabilized.
  \item \textbf{GPU bottleneck:} memory usage increases quadratically with patch overlap.
\end{itemize}

\vspace{3mm}
\begin{block}{Next Steps}
\begin{enumerate}
  \item Implement randomized scan ordering (ePIE style).
  \item Add padding scheme for FFT acceleration.
  \item Start integrating real experimental data from CLS dataset.
\end{enumerate}
\end{block}
\end{frame}

%============================================================
\section{Future Plan}
%============================================================
\begin{frame}{Plans for Next Week}
\footnotesize
\begin{itemize}
  \item [\ding{43}] Finish GPU version of mPIE with adaptive $\beta$.
  \item [\ding{43}] Start drafting short paper for internal progress review.
  \item [\ding{43}] Prepare new figures for PIE-family convergence comparison.
\end{itemize}


\end{frame}

%============================================================
\section*{Acknowledgement}
%============================================================
\begin{frame}[plain]
\centering
\vspace{1cm}
\Huge Thank You!\\[3mm]
\large Questions and Discussion
\end{frame}

\end{document}
